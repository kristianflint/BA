\documentclass[11pt,a4paper]{article} 
\usepackage[danish]{babel}
\usepackage[utf8]{inputenc}

\begin{document}

\section{Forside}

\section{Problemformulering}

\section{Indledning}


\section{Hårde skygger}
Beskriver forskellige måder at ligge hårde skygger på en scene. De 2 metoder der idag bruges til at ligge 

shadowmapping
 shadow volumes

\section{Shadowmap}

\subsection{Teori}
Beskrivelse af shadow mapping teknikken. heriblandt fordele og ulemper.

\subsection{Pratisk}
Beskriver problemer der opstår når shadow mapping teknikken implementeres.
 
\subsubsection{shadow acne}
Beskriver  baising og hvilke problemer dette løser i forhold til den praktiske implementering af shadow mapping. Og hvilke problemer det også kan skabe (peter panning)
\subsubsection{peter panning}

\subsection{Forbedringer/aliasing}
Kigger på mulighederne for at forbedre shadow mapping teknikken vha. aliasing:

forskellige metoder:
Interpolated shadowing
Percentage Closer Filtering (PCF)
Variance Shadow Mapping (VSM)


\section{Stencil Shadow Volumes}

\subsection{Teori}
Beskrivelse af Stencil shadow Volumes teknikken. heriblandt fordele og ulemper.


\subsection{Depth-Pass Vs. Depth-Fail}
Sammen ligner de 2 metoder Depth-Pass og Depth-Fail for at bestemme om en vertex er i skygge eller ej.

evt kun meget kort om Depth-Pass


\section{Afprøvning}
 De 2 metoder afprøves, og det vises hvordan de forskellige metoder og filtre påvirker resultaterne.	

\subsection{Køretider}
Køretiderner for de 2 metoder afprøves.

\section{Konklution}



\end{document}